
\documentclass[11pt,a4paper]{article}

\usepackage[T1]{fontenc}
\usepackage[utf8]{inputenc}
\usepackage[french]{babel}
\usepackage{lmodern}
\usepackage{geometry}
\geometry{margin=2.2cm}

\usepackage{hyperref}
\usepackage{xcolor}
\usepackage{graphicx}
\usepackage{booktabs}
\usepackage{longtable}
\usepackage{array}
\usepackage{enumitem}
\usepackage{amsmath}
\usepackage{amssymb}

\usepackage{listings}
\lstdefinestyle{mcee}{
  basicstyle=\ttfamily\small,
  columns=fullflexible,
  breaklines=true,
  frame=single,
  rulecolor=\color{black!25},
  backgroundcolor=\color{black!2},
  keywordstyle=\color{blue!60!black},
  commentstyle=\color{green!40!black},
  stringstyle=\color{orange!60!black},
  showstringspaces=false,
  tabsize=2
}
\lstset{style=mcee}

\title{MCEE --- Résumé Technique pour le Développement\\\large (avec Système de Phases)}
\author{}
\date{\today}

\begin{document}
\maketitle
\tableofcontents
\newpage

\section{Vue d'ensemble}
Le \textbf{MCEE} (\emph{Modèle Complet d'Évaluation des États}) est un système émotionnel complet intégrant :
\begin{itemize}[leftmargin=1.2em]
  \item \textbf{24 émotions instantanées} (prédites par le module C++) ;
  \item \textbf{Système de phases émotionnelles} (8 phases qui modulent le comportement) ;
  \item \textbf{Graphe de mémoire Neo4j} (souvenirs, concepts, traumas) ;
  \item \textbf{Mécanismes de fusion et modulation} (adaptatifs selon la phase) ;
  \item \textbf{Système d'urgence \textquotedblleft Amyghaleon\textquotedblright} (déclenché selon la phase).
\end{itemize}

\section{Architecture des composants (avec phases)}
\subsection{Schéma d'ensemble}
\begin{lstlisting}[language={},caption={Architecture (ASCII)}]
┌─────────────────┐
│   Capteurs      │ → Données environnementales
│   Feedbacks     │ → Fb_ext (externe), Fb_int (interne)
└────────┬────────┘
         ↓
┌─────────────────┐
│  Module C++     │ → Prédit 24 émotions E_i(t)
│  (emotion)      │    depuis 14 dimensions
└────────┬────────┘
         ↓
┌─────────────────────────────────────────────────────┐
│  🎭 Phase Detector                                  │
│  Détecte la phase émotionnelle actuelle             │
│                                                      │
│  Input:  24 émotions                                │
│  Output: Phase + PhaseConfig (α,β,γ,δ,θ,seuils)      │
│                                                      │
│  Phases: SERENITE | JOIE | EXPLORATION | ANXIETE     │
│          PEUR | TRISTESSE | DEGOUT | CONFUSION       │
└────────┬────────────────────────────────────────────┘
         ↓
┌─────────────────┐
│  MCEE Engine    │ → Mise à jour E_i(t+1) avec coefficients de phase
│  (Python)       │    Fusion → E_global(t+1)
│                 │    Applique les paramètres de la phase active
└────────┬────────┘
         ↓
┌─────────────────┐
│  Neo4j Graphe   │ → Souvenirs, concepts, traumas
│  Mémoire        │    Activation, oubli, renforcement
│                 │    Consolidation modulée par phase
└────────┬────────┘
         ↓
┌─────────────────┐
│  Amyghaleon     │ → Réactions d'urgence (seuil selon phase)
│  (court-circuit)│    Phase PEUR: seuil = 0.50 ⚠️
│                 │    Phase SERENITE: seuil = 0.85 ✅
└─────────────────┘
\end{lstlisting}

\subsection{Flux de données détaillé}
\begin{enumerate}[leftmargin=1.4em]
  \item Le module C++ émet 24 émotions $\rightarrow$ RabbitMQ (\texttt{mcee.emotional.input}).
  \item Le \emph{Phase Detector} reçoit et analyse $\rightarrow$ détecte la phase actuelle.
  \item Le \emph{Phase Detector} fournit \texttt{PhaseConfig} $\rightarrow$ coefficients adaptés.
  \item Le \emph{MCEE Engine} reçoit :
  \begin{itemize}[leftmargin=1.2em]
    \item les 24 émotions brutes ;
    \item la phase actuelle ;
    \item les coefficients $(\alpha,\beta,\gamma,\delta,\theta)$ de cette phase ;
    \item les seuils (Amyghaleon, consolidation, etc.).
  \end{itemize}
  \item Le \emph{MCEE Engine} applique les formules avec les coefficients de phase.
  \item Neo4j est mis à jour selon les paramètres de phase.
  \item Amyghaleon vérifie le seuil de la phase actuelle.
\end{enumerate}

\section{Système de phases émotionnelles}
\subsection{Les 8 phases}
\renewcommand{\arraystretch}{1.15}
\begin{longtable}{>{\bfseries}p{2.6cm} p{1.2cm} p{3.2cm} p{0.9cm} p{0.9cm} p{0.9cm} p{0.9cm} p{1.6cm} p{4.2cm}}
\toprule
Phase & Prio. & Trigger & $\alpha$ & $\delta$ & $\gamma$ & $\theta$ & Seuil Amyg. & Caractéristiques\\
\midrule
\endfirsthead
\toprule
Phase & Prio. & Trigger & $\alpha$ & $\delta$ & $\gamma$ & $\theta$ & Seuil Amyg. & Caractéristiques\\
\midrule
\endhead
SÉRÉNITÉ & 1 & Calme$>0.5$ & 0.25 & 0.30 & 0.12 & 0.10 & 0.85 & Équilibre, apprentissage optimal\\
JOIE & 2 & Joie$>0.6$ & 0.40 & 0.35 & 0.08 & 0.05 & 0.95 & Euphorie, renforcement positif\\
EXPLORATION & 2 & Intérêt$>0.6$ & 0.35 & 0.25 & 0.10 & 0.15 & 0.80 & Apprentissage maximal\\
ANXIÉTÉ & 3 & Anxiété $0.5$--$0.8$ & 0.40 & 0.45 & 0.06 & 0.08 & 0.70 & Hypervigilance, biais négatif\\
PEUR & 5 & Peur$>0.8$ & 0.60 & 0.70 & 0.02 & 0.02 & 0.50 & \textbf{URGENCE} --- Traumas dominants\\
TRISTESSE & 3 & Tristesse$>0.6$ & 0.20 & 0.55 & 0.05 & 0.12 & 0.90 & Rumination, introspection\\
DÉGOÛT & 4 & Dégoût$>0.6$ & 0.50 & 0.40 & 0.08 & 0.08 & 0.75 & Évitement, associations négatives\\
CONFUSION & 2 & Confusion$>0.6$ & 0.35 & 0.50 & 0.15 & 0.15 & 0.80 & Recherche d'info, incertitude\\
\bottomrule
\end{longtable}

\subsection{Transitions entre phases}
\textbf{Règles :}
\begin{enumerate}[leftmargin=1.4em]
  \item \textbf{Priorité} : la phase \textbf{PEUR} court-circuite toutes les autres.
  \item \textbf{Hystérésis} : marge de 0.15 pour éviter les oscillations.
  \item \textbf{Durée minimale} : 30 secondes avant changement (configurable).
  \item \textbf{Urgence} : Peur $> 0.85$ \textbf{ou} Horreur $> 0.8$ $\rightarrow$ transition immédiate.
\end{enumerate}

\textbf{Exemple de séquence :}
\begin{lstlisting}[language={}]
SÉRÉNITÉ (120s) → EXPLORATION (45s) → JOIE (60s) → ANXIÉTÉ (35s) → PEUR (15s)
                                                                        ↓
                                                              AMYGHALEON ACTIVÉ
\end{lstlisting}

\section{Formules clés à implémenter (avec phases)}
\subsection{Mise à jour des émotions individuelles}
\[
E_i(t+1) = E_i(t) + \alpha_{\text{phase}}\cdot Fb_{\text{ext}} + \beta_{\text{phase}}\cdot Fb_{\text{int}}(t)
- \gamma_{\text{phase}}\cdot \Delta t + \delta_{\text{phase}}\cdot Influence_{\text{Souvenirs}} + \theta_{\text{phase}}\cdot W_t
\]

\subsection{Extrait d'implémentation (Python)}
\begin{lstlisting}[language=Python]
class EmotionUpdater:
    def __init__(self):
        self.alpha = 0.3
        self.beta  = 0.2
        self.gamma = 0.1
        self.delta = 0.4
        self.theta = 0.1

    def set_coefficients_from_phase(self, phase_config):
        self.alpha = phase_config['alpha']
        self.beta  = phase_config['beta']
        self.gamma = phase_config['gamma']
        self.delta = phase_config['delta']
        self.theta = phase_config['theta']

    def update_emotion(self, E_current, fb_ext, fb_int, delta_t,
                       influence_memories, wisdom):
        E_next = (E_current +
                  self.alpha * fb_ext +
                  self.beta  * fb_int -
                  self.gamma * delta_t +
                  self.delta * influence_memories +
                  self.theta * wisdom)
        return max(0.0, min(1.0, E_next))
\end{lstlisting}

\subsection{Variance (détection d'anomalies)}
\[
Var_i(t) = \frac{1}{m}\sum_{j=1}^{m}\left[E_i(t) - S_{i,j}\right]^2
\]

\subsection{Fusion des émotions (stabilisation \texttt{tanh})}
\[
E_{\text{global}}(t+1) = \tanh\Bigl(E_{\text{global}}(t) + \sum_i \bigl[E_i(t+1)\cdot (1 - Var_{\text{global}}(t))\bigr]\Bigr)
\]

\section{Gestion de la mémoire (Neo4j) --- modulée par phase}
\subsection{Structure des nœuds}
\begin{lstlisting}[language=SQL,caption={Exemple Cypher}]
CREATE (s:Souvenir {
    name: 'Événement X',
    date: date('2025-12-19'),
    emotions: [0.7, 0.2, ...],
    dominant: 'Joie',
    valence: 0.7,
    intensity: 0.8,
    last_activated: datetime(),
    activation_count: 1,
    weight: 0.5,
    type: 'positif',
    state: 'SouvenirConsolider',
    phase_at_creation: 'JOIE'
})
\end{lstlisting}

\subsection{Activation des souvenirs (rappel)}
\begin{lstlisting}[language=Python]
A(S_i) = forget(S_i, t) * (1 + R(S_i)) *
         Σ[ C(S_i, S_k) * Me(S_i, E_current) * U(S_i) ]
\end{lstlisting}

\subsection{Requêtes adaptées selon phase}
\begin{lstlisting}[language=Python]
def query_relevant_memories(phase, emotions):
    if phase == 'PEUR':
        query = """
        MATCH (s:Souvenir)
        WHERE s.dominant IN ['Peur', 'Horreur', 'Anxiété']
           OR EXISTS((s)<-[:CONCERNE]-(t:Trauma))
        RETURN s
        ORDER BY s.intensity DESC, s.weight DESC
        LIMIT 20
        """
    elif phase == 'JOIE':
        query = """
        MATCH (s:Souvenir)
        WHERE s.valence > 0.5
          AND s.dominant IN ['Joie', 'Satisfaction', 'Excitation']
        RETURN s
        ORDER BY s.valence DESC
        LIMIT 10
        """
    else:
        query = """
        MATCH (s:Souvenir)
        WHERE s.last_activated > datetime() - duration({days: 30})
        RETURN s
        ORDER BY s.weight DESC
        LIMIT 10
        """
    return neo4j.run(query)
\end{lstlisting}

\section{Système d'urgence \textquotedblleft Amyghaleon\textquotedblright}
\subsection{Seuils par phase}
\begin{lstlisting}[language=Python]
AMYGHALEON_THRESHOLDS = {
    'SERENITE':    0.85,
    'JOIE':        0.95,
    'EXPLORATION': 0.80,
    'ANXIETE':     0.70,
    'PEUR':        0.50,
    'TRISTESSE':   0.90,
    'DEGOUT':      0.75,
    'CONFUSION':   0.80
}
\end{lstlisting}

\subsection{Exemple : PEUR vs SÉRÉNITÉ}
\begin{lstlisting}[language=Python]
# Situation identique : Peur = 0.65

# Phase SÉRÉNITÉ (seuil = 0.85)
if 0.65 > 0.85:  # False
    pass

# Phase PEUR (seuil = 0.50)
if 0.65 > 0.50:  # True
    # URGENCE détectée, court-circuit
    pass
\end{lstlisting}

\section{Annexe : Tableau complet des phases}
\renewcommand{\arraystretch}{1.15}
\begin{longtable}{>{\bfseries}p{2.6cm} p{0.9cm} p{0.9cm} p{0.9cm} p{0.9cm} p{0.9cm} p{1.6cm} p{1.5cm} p{1.2cm} p{1.2cm} p{4.0cm}}
\toprule
Phase & $\alpha$ & $\beta$ & $\gamma$ & $\delta$ & $\theta$ & Seuil Amyg. & Consolid. & Learn & Focus & Comportement\\
\midrule
\endfirsthead
\toprule
Phase & $\alpha$ & $\beta$ & $\gamma$ & $\delta$ & $\theta$ & Seuil Amyg. & Consolid. & Learn & Focus & Comportement\\
\midrule
\endhead
SÉRÉNITÉ & 0.25 & 0.15 & 0.12 & 0.30 & 0.10 & 0.85 & 0.4 & 1.0 & 0.5 & Équilibre, apprentissage optimal\\
JOIE & 0.40 & 0.25 & 0.08 & 0.35 & 0.05 & 0.95 & 0.5 & 1.3 & 0.3 & Renforcement positif, risque sous-estimé\\
EXPLORATION & 0.35 & 0.10 & 0.10 & 0.25 & 0.15 & 0.80 & 0.6 & 1.5 & 0.8 & Apprentissage maximal, attention focalisée\\
ANXIÉTÉ & 0.40 & 0.30 & 0.06 & 0.45 & 0.08 & 0.70 & 0.4 & 0.8 & 0.6 & Hypervigilance, biais négatif\\
PEUR & 0.60 & 0.45 & 0.02 & 0.70 & 0.02 & 0.50 & 0.8 & 0.3 & 0.95 & URGENCE, traumas dominants\\
TRISTESSE & 0.20 & 0.40 & 0.05 & 0.55 & 0.12 & 0.90 & 0.5 & 0.6 & 0.4 & Rumination, introspection\\
DÉGOÛT & 0.50 & 0.25 & 0.08 & 0.40 & 0.08 & 0.75 & 0.6 & 0.9 & 0.7 & Évitement, associations négatives\\
CONFUSION & 0.35 & 0.30 & 0.15 & 0.50 & 0.15 & 0.80 & 0.3 & 0.7 & 0.5 & Recherche d'info, incertitude\\
\bottomrule
\end{longtable}

\end{document}
