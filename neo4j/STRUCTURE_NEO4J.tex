\documentclass[11pt,a4paper]{article}

% ============================================================================
% PACKAGES
% ============================================================================
\usepackage[utf8]{inputenc}
\usepackage[T1]{fontenc}
\usepackage[french]{babel}
\usepackage{geometry}
\usepackage{graphicx}
\usepackage{xcolor}
\usepackage{listings}
\usepackage{booktabs}
\usepackage{longtable}
\usepackage{array}
\usepackage{hyperref}
\usepackage{fancyhdr}
\usepackage{titlesec}
\usepackage{tcolorbox}
\usepackage{tikz}
\usetikzlibrary{shapes.geometric, arrows.meta, positioning, fit, backgrounds}

% ============================================================================
% CONFIGURATION
% ============================================================================
\geometry{margin=2.5cm}
\hypersetup{
    colorlinks=true,
    linkcolor=blue!60!black,
    urlcolor=blue!60!black,
    citecolor=blue!60!black
}

% Couleurs personnalisées
\definecolor{codebackground}{RGB}{245, 245, 245}
\definecolor{codeborder}{RGB}{200, 200, 200}
\definecolor{cypherkey}{RGB}{0, 102, 153}
\definecolor{cypherstring}{RGB}{163, 21, 21}
\definecolor{cyphercomment}{RGB}{0, 128, 0}
\definecolor{nodecolor}{RGB}{66, 133, 244}
\definecolor{relationcolor}{RGB}{234, 67, 53}
\definecolor{memorycolor}{RGB}{52, 168, 83}
\definecolor{conceptcolor}{RGB}{251, 188, 5}
\definecolor{sessioncolor}{RGB}{154, 160, 166}

% Configuration listings pour Cypher
\lstdefinelanguage{Cypher}{
    keywords={MATCH, CREATE, MERGE, SET, RETURN, WHERE, WITH, ORDER, BY, LIMIT, DELETE, REMOVE, ON, AND, OR, NOT, AS, IN, IS, NULL, TRUE, FALSE, CASE, WHEN, THEN, ELSE, END, OPTIONAL, UNWIND, FOREACH, CALL, YIELD, DETACH, EXISTS, COUNT, SUM, AVG, MIN, MAX, COLLECT, DISTINCT, DESC, ASC, INDEX, FOR, IF},
    keywordstyle=\color{cypherkey}\bfseries,
    sensitive=true,
    morecomment=[l]{//},
    morecomment=[s]{/*}{*/},
    commentstyle=\color{cyphercomment}\itshape,
    stringstyle=\color{cypherstring},
    morestring=[b]',
    morestring=[b]"
}

\lstset{
    language=Cypher,
    backgroundcolor=\color{codebackground},
    frame=single,
    rulecolor=\color{codeborder},
    basicstyle=\ttfamily\small,
    breaklines=true,
    breakatwhitespace=true,
    tabsize=4,
    showstringspaces=false,
    numbers=none,
    xleftmargin=0.5cm,
    xrightmargin=0.5cm,
    aboveskip=0.5cm,
    belowskip=0.5cm
}

% Configuration pour JSON
\lstdefinelanguage{JSON}{
    morestring=[b]",
    stringstyle=\color{cypherstring},
    literate=
        *{:}{{{\color{black}:}}}{1}
        {,}{{{\color{black},}}}{1}
        {\{}{{{\color{black}\{}}}{1}
        {\}}{{{\color{black}\}}}}{1}
        {[}{{{\color{black}[}}}{1}
        {]}{{{\color{black}]}}}{1}
}

% En-têtes et pieds de page
\pagestyle{fancy}
\fancyhf{}
\fancyhead[L]{\leftmark}
\fancyhead[R]{MCEE - Neo4j}
\fancyfoot[C]{\thepage}
\renewcommand{\headrulewidth}{0.4pt}

% Formatage des titres
\titleformat{\section}
    {\normalfont\Large\bfseries\color{nodecolor}}
    {\thesection}{1em}{}
\titleformat{\subsection}
    {\normalfont\large\bfseries\color{nodecolor!80!black}}
    {\thesubsection}{1em}{}
\titleformat{\subsubsection}
    {\normalfont\normalsize\bfseries\color{nodecolor!60!black}}
    {\thesubsubsection}{1em}{}

% Boîtes colorées
\tcbuselibrary{skins, breakable}

\newtcolorbox{notebox}{
    colback=blue!5,
    colframe=blue!50!black,
    fonttitle=\bfseries,
    title={Note:},
    breakable
}

\newtcolorbox{warningbox}{
    colback=orange!5,
    colframe=orange!50!black,
    fonttitle=\bfseries,
    title={Important:},
    breakable
}

\newtcolorbox{nodebox}[1]{
    colback=nodecolor!5,
    colframe=nodecolor,
    fonttitle=\bfseries,
    title={#1},
    breakable
}

% ============================================================================
% DOCUMENT
% ============================================================================
\begin{document}

% Page de titre
\begin{titlepage}
    \centering
    \vspace*{2cm}
    
    {\Huge\bfseries Structure Neo4j\par}
    \vspace{0.5cm}
    {\Large\itshape MCEE - Modèle Complet d'Évaluation des États\par}
    
    \vspace{2cm}
    
    \begin{tikzpicture}[
        node distance=2cm,
        every node/.style={font=\sffamily},
        memory/.style={rectangle, rounded corners, draw=memorycolor, fill=memorycolor!20, minimum width=3cm, minimum height=1cm, thick},
        concept/.style={rectangle, rounded corners, draw=conceptcolor, fill=conceptcolor!20, minimum width=2.5cm, minimum height=1cm, thick},
        session/.style={rectangle, rounded corners, draw=sessioncolor, fill=sessioncolor!20, minimum width=2.5cm, minimum height=1cm, thick},
        relation/.style={-{Stealth[length=3mm]}, thick, relationcolor}
    ]
        \node[memory] (mem) {Memory};
        \node[concept, right=of mem] (con1) {Concept};
        \node[concept, right=of con1] (con2) {Concept};
        \node[session, below=of mem] (ses) {Session};
        
        \draw[relation] (mem) -- node[above, font=\small] {EVOQUE} (con1);
        \draw[relation] (con1) -- node[above, font=\small] {SEMANTIQUE} (con2);
        \draw[relation] (ses) -- node[left, font=\small] {CONTAINS} (mem);
    \end{tikzpicture}
    
    \vspace{2cm}
    
    {\large Version 3.1\par}
    {\large Architecture simplifiée - 24 émotions\par}
    
    \vfill
    
    {\large Documentation technique\par}
    {\large \today\par}
\end{titlepage}

% Table des matières
\tableofcontents
\newpage

% ============================================================================
\section{Vue d'ensemble}
% ============================================================================

Le système utilise Neo4j comme base de données graphe pour stocker les mémoires émotionnelles, les concepts sémantiques et leurs relations.

\subsection{Philosophie de conception}

L'architecture v3.1 abandonne les patterns statiques (SERENITE, ANXIETE, etc.) au profit d'un système basé sur \textbf{24 émotions continues}. Cette approche permet :

\begin{itemize}
    \item \textbf{Nuances émotionnelles} -- La peur à 0.3 + anxiété à 0.4 vs terreur à 0.9 + horreur à 0.8
    \item \textbf{Combinaisons naturelles} -- Joie + nostalgie + admiration simultanées
    \item \textbf{Évolution temporelle} -- Suivi des changements par \texttt{sentence\_id}
\end{itemize}

\subsection{Architecture mémoire}

\begin{center}
\begin{tikzpicture}[
    node distance=1.5cm and 2.5cm,
    every node/.style={font=\sffamily\small},
    memory/.style={rectangle, rounded corners, draw=memorycolor, fill=memorycolor!20, minimum width=2.2cm, minimum height=0.8cm, thick},
    concept/.style={rectangle, rounded corners, draw=conceptcolor, fill=conceptcolor!20, minimum width=2.2cm, minimum height=0.8cm, thick},
    session/.style={rectangle, rounded corners, draw=sessioncolor, fill=sessioncolor!20, minimum width=2.2cm, minimum height=0.8cm, thick},
    relation/.style={-{Stealth[length=2.5mm]}, thick},
    label/.style={font=\sffamily\scriptsize, above, sloped}
]
    % Nodes
    \node[memory] (mem) {Memory};
    \node[concept, right=of mem] (con1) {Concept};
    \node[concept, right=of con1] (con2) {Concept};
    \node[session, below=of mem] (ses) {Session};
    \node[memory, below=of ses] (trauma) {Trauma};
    
    % Relations
    \draw[relation, memorycolor] (mem) -- node[label] {EVOQUE} (con1);
    \draw[relation, conceptcolor] (con1) -- node[label] {SEMANTIQUE} (con2);
    \draw[relation, sessioncolor] (ses) -- node[label, left] {CONTAINS} (mem);
    \draw[relation, red!70] (trauma) -- node[label, below] {TRIGGERS} (con1);
\end{tikzpicture}
\end{center}

% ============================================================================
\section{Nœuds (Nodes)}
% ============================================================================

\subsection{Memory}

Représente un souvenir/mémoire avec son contexte émotionnel complet.

\begin{nodebox}{Memory Node}
\begin{lstlisting}
(:Memory {
    id: String,                    -- Identifiant unique
    
    -- Emotions (24 dimensions)
    emotional_states: String,      -- JSON: {"sentence_id": [24 floats]}
    dominant: String,              -- Emotion dominante
    intensity: Float,              -- Intensite [0.0 - 1.0]
    valence: Float,                -- Valence [-1.0 a 1.0]
    
    -- Metadonnees
    weight: Float,                 -- Poids/importance [0.0 - 1.0]
    context: String,               -- Texte du contexte
    category: String,              -- Categorie semantique
    keywords: [String],            -- Mots-cles extraits
    
    -- Classification memoire
    type: String,                  -- "MCT" | "MLT" | "Episodic" | "Semantic"
    consolidated: Boolean,         -- True si consolide en MLT
    active: Boolean,               -- True si actif
    
    -- Statistiques
    activation_count: Integer,     -- Nombre de reactivations
    
    -- Timestamps
    created_at: DateTime,
    updated_at: DateTime,
    last_accessed: DateTime
})
\end{lstlisting}
\end{nodebox}

\subsubsection{Labels additionnels}

\begin{itemize}
    \item \texttt{:Trauma} -- Pour les mémoires traumatiques (weight $\geq$ 0.3)
    \item \texttt{:MCT} -- Mémoire à Court Terme
    \item \texttt{:MLT} -- Mémoire à Long Terme
    \item \texttt{:Archived} -- Mémoire archivée
\end{itemize}

\subsubsection{Format emotional\_states}

Les états émotionnels sont stockés en JSON string avec le \texttt{sentence\_id} comme clé :

\begin{lstlisting}[language=JSON]
{
    "1": [0.8, 0.24, 0.16, 0.0, 0.0, ...],
    "4": [0.1, 0.7, 0.21, 0.0, 0.0, ...],
    "7": [0.0, 0.0, 0.0, 0.9, 0.27, ...]
}
\end{lstlisting}

\subsubsection{Les 24 émotions}

\begin{center}
\begin{tabular}{clclcl}
\toprule
\textbf{Index} & \textbf{Émotion} & \textbf{Index} & \textbf{Émotion} & \textbf{Index} & \textbf{Émotion} \\
\midrule
0 & Joie & 8 & Calme & 16 & Triomphe \\
1 & Confiance & 9 & Intérêt & 17 & Admiration \\
2 & Peur & 10 & Fascination & 18 & Adoration \\
3 & Surprise & 11 & Confusion & 19 & Amusement \\
4 & Tristesse & 12 & Ennui & 20 & Anxiété \\
5 & Dégoût & 13 & Gêne & 21 & Émerveillement \\
6 & Horreur & 14 & Excitation & 22 & Nostalgie \\
7 & Douleur emp. & 15 & Soulagement & 23 & Satisfaction \\
\bottomrule
\end{tabular}
\end{center}

\begin{notebox}
Les 24 émotions permettent de capturer des nuances fines. Par exemple, "peur légère" = Peur(0.3) + Anxiété(0.4), tandis que "terreur" = Peur(0.9) + Horreur(0.8) + Tristesse(0.5).
\end{notebox}

\subsection{Trauma}

Extension de Memory pour les souvenirs traumatiques.

\begin{nodebox}{Trauma Node}
\begin{lstlisting}
(:Memory:Trauma {
    -- Herite de toutes les proprietes Memory
    id: String,
    emotional_states: String,
    ...
    
    -- Proprietes specifiques Trauma
    trauma: Boolean,               -- Toujours true
    triggers: [String],            -- Mots/concepts declencheurs
    protection_level: Float        -- Niveau de protection [0.0 - 1.0]
    
    -- Note: weight a un plancher (floor) de 0.3
})
\end{lstlisting}
\end{nodebox}

\subsection{Concept}

Représente un concept sémantique extrait des mémoires.

\begin{nodebox}{Concept Node}
\begin{lstlisting}
(:Concept {
    name: String,                  -- Nom du concept (minuscules)
    
    -- Emotions accumulees
    emotional_states: String,      -- JSON: {"sentence_id": [24 floats]}
    
    -- References
    memory_ids: [String],          -- Liste des IDs de memoires liees
    
    -- Timestamps
    created_at: DateTime,
    updated_at: DateTime
})
\end{lstlisting}
\end{nodebox}

\begin{notebox}
Les \texttt{emotional\_states} d'un Concept accumulent les états de toutes les mémoires qui l'évoquent, permettant une analyse de l'évolution émotionnelle du concept dans le temps.
\end{notebox}

\subsection{Session}

Représente une session d'interaction.

\begin{nodebox}{Session Node}
\begin{lstlisting}
(:Session:MCT {
    id: String,                    -- Identifiant unique
    
    -- Metriques emotionnelles
    stability: Float,              -- Stabilite [0.0 - 1.0]
    volatility: Float,             -- Volatilite [0.0 - 1.0]
    trend: String,                 -- "stable" | "ascending" | "descending"
    state_count: Integer,          -- Nombre d'etats enregistres
    
    -- Timestamps
    created_at: DateTime,
    updated_at: DateTime
})
\end{lstlisting}
\end{nodebox}

\subsection{EmotionalState}

État émotionnel ponctuel (lié à une Session).

\begin{nodebox}{EmotionalState Node}
\begin{lstlisting}
(:EmotionalState {
    emotions: [Float],             -- 24 valeurs d'emotions
    dominant: String,              -- Emotion dominante
    intensity: Float,              -- Intensite
    valence: Float,                -- Valence
    timestamp: DateTime            -- Moment de l'enregistrement
})
\end{lstlisting}
\end{nodebox}

% ============================================================================
\section{Relations}
% ============================================================================

\subsection{EVOQUE}

Lie une mémoire aux concepts qu'elle évoque.

\begin{lstlisting}
(:Memory)-[:EVOQUE]->(:Concept)
\end{lstlisting}

\textbf{Propriétés :} Aucune (relation simple)

\subsection{SEMANTIQUE}

Relation sémantique entre deux concepts.

\begin{lstlisting}
(:Concept)-[:SEMANTIQUE {
    type: String,                  -- Type de relation
    count: Integer,                -- Nombre d'occurrences
    memory_ids: [String],          -- Memoires source
    emotional_states: String       -- JSON des etats emotionnels
}]->(:Concept)
\end{lstlisting}

\subsubsection{Types de relations sémantiques}

\begin{center}
\begin{tabular}{lll}
\toprule
\textbf{Type} & \textbf{Description} & \textbf{Exemple} \\
\midrule
\texttt{EST} & Attribution/Identité & "chat" $\rightarrow$ "noir" \\
\texttt{UTILISE} & Action/Utilisation & "chat" $\rightarrow$ "dort" \\
\texttt{LOCALISE} & Localisation & "chat" $\rightarrow$ "canapé" \\
\texttt{MODIFIE} & Modification/Adverbe & "paisiblement" $\rightarrow$ "dort" \\
\texttt{POSSEDE} & Possession & "marie" $\rightarrow$ "chat" \\
\texttt{CAUSE} & Causalité & "pluie" $\rightarrow$ "tristesse" \\
\bottomrule
\end{tabular}
\end{center}

\subsection{ASSOCIE\_A}

Association générique entre Memory et Concept.

\begin{lstlisting}
(:Memory)-[:ASSOCIE_A {
    strength: Float,               -- Force de l'association [0.0 - 1.0]
    created_at: DateTime
}]->(:Concept)
\end{lstlisting}

\subsection{CONTAINS}

Lie une session à ses états émotionnels.

\begin{lstlisting}
(:Session)-[:CONTAINS]->(:EmotionalState)
\end{lstlisting}

\subsection{TRIGGERS}

Lie un trauma à ses concepts déclencheurs.

\begin{lstlisting}
(:Trauma)-[:TRIGGERS]->(:Concept)
\end{lstlisting}

% ============================================================================
\section{Index recommandés}
% ============================================================================

\begin{lstlisting}
-- Index primaires
CREATE INDEX memory_id IF NOT EXISTS FOR (m:Memory) ON (m.id);
CREATE INDEX concept_name IF NOT EXISTS FOR (c:Concept) ON (c.name);
CREATE INDEX session_id IF NOT EXISTS FOR (s:Session) ON (s.id);

-- Index de recherche
CREATE INDEX memory_dominant IF NOT EXISTS FOR (m:Memory) ON (m.dominant);
CREATE INDEX memory_type IF NOT EXISTS FOR (m:Memory) ON (m.type);
CREATE INDEX memory_weight IF NOT EXISTS FOR (m:Memory) ON (m.weight);

-- Index composites
CREATE INDEX memory_type_weight IF NOT EXISTS 
    FOR (m:Memory) ON (m.type, m.weight);
\end{lstlisting}

% ============================================================================
\section{Requêtes Cypher courantes}
% ============================================================================

\subsection{Créer une mémoire avec concepts}

\begin{lstlisting}
// 1. Creer la memoire
CREATE (m:Memory:MCT {
    id: $id,
    emotional_states: $emotional_states,  // JSON string
    dominant: $dominant,
    intensity: $intensity,
    valence: $valence,
    weight: $weight,
    type: 'MCT',
    created_at: datetime()
})

// 2. Creer/lier les concepts
MERGE (c:Concept {name: $concept_name})
ON CREATE SET c.created_at = datetime(), c.memory_ids = [$id]
ON MATCH SET c.memory_ids = c.memory_ids + $id
MERGE (m)-[:EVOQUE]->(c)
\end{lstlisting}

\subsection{Rechercher par sentence\_id}

\begin{lstlisting}
// Concepts contenant un sentence_id specifique
MATCH (c:Concept)
WHERE c.emotional_states IS NOT NULL 
  AND c.emotional_states CONTAINS '"<sentence_id>":'
RETURN c
\end{lstlisting}

\subsection{Trouver les mémoires similaires}

\begin{lstlisting}
MATCH (m:Memory)
WHERE m.intensity IS NOT NULL AND m.valence IS NOT NULL
WITH m,
     1 - abs(m.intensity - $query_intensity) AS intensity_sim,
     1 - abs(m.valence - $query_valence) AS valence_sim
WITH m, (intensity_sim + valence_sim) / 2 AS similarity
WHERE similarity >= $threshold
RETURN m ORDER BY similarity DESC LIMIT $limit
\end{lstlisting}

\subsection{Consolider MCT vers MLT}

\begin{lstlisting}
MATCH (m:Memory)
WHERE (m.type = 'MCT' OR m.type IS NULL) 
  AND (m.consolidated IS NULL OR m.consolidated = false)
  AND m.weight >= $threshold
SET m.type = 'MLT', m.consolidated = true, m:MLT
REMOVE m:MCT
RETURN count(m) AS consolidated
\end{lstlisting}

% ============================================================================
\section{Conventions importantes}
% ============================================================================

\subsection{Sérialisation JSON}

\begin{warningbox}
Les \texttt{emotional\_states} sont \textbf{toujours} stockés comme JSON string dans Neo4j.
\end{warningbox}

\begin{lstlisting}[language=Python]
# Python -> Neo4j
json.dumps({str(k): v for k, v in emotional_states.items()})

# Neo4j -> Python  
json.loads(emotional_states_string)
\end{lstlisting}

\subsection{Clés de sentence\_id}

Les clés dans \texttt{emotional\_states} sont \textbf{toujours des strings} :

\begin{lstlisting}[language=JSON]
{"1": [...], "42": [...]}   // Correct
{1: [...], 42: [...]}       // Incorrect
\end{lstlisting}

\subsection{Recherche dans JSON}

Utiliser \texttt{CONTAINS} pour rechercher dans les JSON strings :

\begin{lstlisting}
WHERE c.emotional_states CONTAINS '"42":'  // Cherche sentence_id 42
\end{lstlisting}

\subsection{Syntaxe Neo4j 5+}

\begin{center}
\begin{tabular}{ll}
\toprule
\textbf{Ancienne syntaxe} & \textbf{Neo4j 5+} \\
\midrule
\texttt{NOT EXISTS(m.prop)} & \texttt{m.prop IS NULL} \\
\texttt{EXISTS(m.prop)} & \texttt{m.prop IS NOT NULL} \\
\bottomrule
\end{tabular}
\end{center}

% ============================================================================
\section{Analyse émotionnelle}
% ============================================================================

Lors de la récupération d'un concept avec \texttt{get\_concept}, une analyse est effectuée :

\begin{lstlisting}[language=JSON]
{
    "name": "parc",
    "memory_ids": ["MEM_1", "MEM_2"],
    "emotional_states": {"1": [...], "4": [...]},
    "sentence_ids": ["1", "4"],
    "emotional_analysis": {
        "dominant_emotion": "Joie",
        "average_valence": 0.73,
        "stability": 0.96,
        "trajectory": "stable",
        "trauma_score": 0.12
    }
}
\end{lstlisting}

\begin{center}
\begin{tabular}{ll}
\toprule
\textbf{Champ} & \textbf{Description} \\
\midrule
\texttt{dominant\_emotion} & Émotion la plus fréquente parmi les 24 \\
\texttt{average\_valence} & Valence moyenne [--1.0, 1.0] \\
\texttt{stability} & Stabilité émotionnelle [0.0, 1.0] \\
\texttt{trajectory} & \texttt{"stable"} | \texttt{"ascending"} | \texttt{"descending"} \\
\texttt{trauma\_score} & Score de trauma [0.0, 1.0] \\
\bottomrule
\end{tabular}
\end{center}

% ============================================================================
\section{Schéma visuel complet}
% ============================================================================

\begin{center}
\begin{tikzpicture}[
    node distance=1.2cm and 1.8cm,
    every node/.style={font=\sffamily\footnotesize},
    memorybox/.style={
        rectangle, rounded corners=5pt,
        draw=memorycolor, fill=memorycolor!10,
        minimum width=8cm, minimum height=3cm,
        thick
    },
    conceptbox/.style={
        rectangle, rounded corners=5pt,
        draw=conceptcolor, fill=conceptcolor!10,
        minimum width=8cm, minimum height=1.8cm,
        thick
    },
    relation/.style={-{Stealth[length=2.5mm]}, thick, relationcolor},
    property/.style={font=\ttfamily\scriptsize, align=left}
]

% Memory
\node[memorybox] (memory) at (0, 2) {};
\node[above] at (memory.north) {\textbf{Memory (MCT/MLT)}};
\node[property] at (memory.center) {
    id: "MEM\_123"\\
    emotional\_states: '\{"1": [0.8, 0.2, ...24 vals]\}'\\
    dominant: "Joie"\\
    intensity: 0.85, valence: 0.9\\
    weight: 0.7, type: "MLT"
};

% Concept 1
\node[conceptbox] (concept1) at (0, -1) {};
\node[above] at (concept1.north) {\textbf{Concept}};
\node[property] at (concept1.center) {
    name: "parc"\\
    emotional\_states: '\{"1": [...], "4": [...]\}'\\
    memory\_ids: ["MEM\_123", "MEM\_456"]
};

% Concept 2
\node[conceptbox] (concept2) at (0, -3.5) {};
\node[above] at (concept2.north) {\textbf{Concept}};
\node[property] at (concept2.center) {
    name: "canards"\\
    emotional\_states: '\{"1": [...]\}'\\
    memory\_ids: ["MEM\_123"]
};

% Relations
\draw[relation] (memory) -- node[right, font=\scriptsize] {EVOQUE} (concept1);
\draw[relation] (concept1) -- node[right, font=\scriptsize] {SEMANTIQUE \{type: "LOCALISE"\}} (concept2);

\end{tikzpicture}
\end{center}

% ============================================================================
\section{Comparaison avec l'ancienne architecture}
% ============================================================================

\begin{center}
\begin{tabular}{lcc}
\toprule
\textbf{Élément} & \textbf{v3.0 (Patterns)} & \textbf{v3.1 (24 émotions)} \\
\midrule
Patterns statiques & Oui (8 patterns) & Non \\
Transitions & TRANSITION\_TO & Implicites \\
Nuances émotionnelles & Limitées & 24 dimensions \\
Combinaisons & Pattern unique & Multi-émotions \\
Complexité & Plus élevée & Simplifiée \\
Nœuds & Memory, Concept, Pattern, Session & Memory, Concept, Session \\
\bottomrule
\end{tabular}
\end{center}

% ============================================================================
% FIN DU DOCUMENT
% ============================================================================

\vfill
\begin{center}
\rule{0.5\textwidth}{0.4pt}\\[0.5em]
{\small Documentation générée le \today{} -- Version 3.1 (architecture simplifiée, 24 émotions)}
\end{center}

\end{document}
